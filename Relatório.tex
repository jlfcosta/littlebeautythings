\documentclass[12pt,letterpaper]{article}

\usepackage[brazilian]{babel}
\usepackage[utf8]{inputenc}
\usepackage[T1]{fontenc}

\usepackage{fullpage}
\usepackage[top=2cm, bottom=4.5cm, left=2.5cm, right=2.5cm]{geometry}
\usepackage{amsmath,amsthm,amsfonts,amssymb,amscd}
\usepackage{lastpage}
\usepackage{enumerate}
\usepackage{fancyhdr}
\usepackage{mathrsfs}
\usepackage{xcolor}
\usepackage{graphicx}
\usepackage{listings}
\usepackage{hyperref}

\hypersetup{%
  colorlinks=true,
  linkcolor=blue,
  linkbordercolor={0 0 1}
}
 
\renewcommand\lstlistingname{Algorithm}
\renewcommand\lstlistlistingname{Algorithms}
\def\lstlistingautorefname{Alg.}

 
\usepackage{listings}
\usepackage{xcolor}
 
\definecolor{codegreen}{rgb}{0,0.6,0}
\definecolor{codegray}{rgb}{0.5,0.5,0.5}
\definecolor{codepurple}{rgb}{0.58,0,0.82}
\definecolor{codemagenta}{rgb}{0.95,0,0.95}
\definecolor{backcolour}{rgb}{0.95,0.95,0.92}
 
\lstdefinestyle{mystyle}{
    backgroundcolor=\color{backcolour},   
    commentstyle=\color{codegreen},
    keywordstyle=\color{magenta},
    numberstyle=\tiny\color{codegray},
    stringstyle=\color{codepurple},
    basicstyle=\ttfamily\footnotesize,
    breakatwhitespace=false,         
    breaklines=true,                 
    captionpos=b,                    
    keepspaces=true,                 
    numbers=left,                    
    numbersep=5pt,                  
    showspaces=false,                
    showstringspaces=false,
    showtabs=false,                  
    tabsize=2
}
 
\lstset{style=mystyle}
\setlength{\parindent}{0.0in}
\setlength{\parskip}{0.05in}

% Edit these as appropriate
\newcommand\course{Rener Oliveira e Jorge Luiz}
\newcommand\hwnumber{1}                  % <-- homework number
\newcommand\NetIDa{netid19823}           % <-- NetID of person #1
\newcommand\NetIDb{netid12038}           % <-- NetID of person #2 (Comment this line out for problem sets)

%\pagestyle{fancyplain}
%\headheight 35pt              % <-- Comment this line out for problem sets (make sure you are person #1)

%\rhead{\course \\ \today}
%\lfoot{}
%\cfoot{}
%\rfoot{\small\thepage}
%\headsep 1.5em

\title{Relatório - Modelagem de Fenômeno Físicos}
\author{Jorge Luís e Rener Oliveira}

\begin{document}

\maketitle

\section*{Descrição Geral}

O objetivo do projeto, foi construir um jogo, implementado no módulo Python do Processing, que mistura a diversão de uma "Guerra de Lançadores de Mísseis" com conceitos de cinemática vetorial, método e euler e modelagem de resistência do ar.

O jogador é um representado por um círculo no canto inferior esquerdo da tela. No decorrer do jogo, surgiram do topo da tela, pequenos outros círculos, que são os lançadores inimigos que devem ser combatidos. O usuário mira com o mouse aperta a barra de espaços, definindo uma direção na qual seu míssil seguirá para conseguir acertar os inimigos, todos os objetos sofrem com a aceleração da gravidade e a bola principal sobre também com a força de resistência do ar. 

Com isso, se a bola principal atingir os limites de tela laterais e inferior, ela retorna a sua posição inicial; No momento em que ela atinge uma bola inimiga, esta desaparece, e a principal retorna novamente para o começo. Todas essas colisões são contadas internamente para atualizar a Fase do Jogo, que quanto mais avançada, maior o nível de dificuldade. Em alguns níveis surge um obstáculo a mais que é um vento de natureza aleatória que pode ajudar ou atrapalhar a jogabilidade.

\section*{Conceitos Físicos e Implementação}
\subsection*{Lançamento de Projétil com resistência do ar}
O primeiro conceito abordado é a modelagem de lançamento de projéteis considerando a resistência do ar, que será a modelagem da bola principal. Este modelo sugere que as componentes do vetor da Força de Resistência do Ar ($\vec{F}$) são proporcionais à velocidade do objeto. Pela Segunda Lei de Newton, o que temos é:
$$\vec{F}_r=m\cdot\vec{a}=k\vec{v}$$

Ou seja, $\vec{a}=\displaystyle\frac k m \vec{v}$

A constante $k$ é dada por $-\displaystyle\frac12\rho A C_r$ onde:

\begin{itemize}
    \item $\rho$ é a densidade do fluido, que no caso é o ar, que para simplificar, arredondamos para 1;
    \item $A$ é a área de contato do objeto com o fluido. No caso da nossa bolinha, considerando que ela representa uma esfera de mesmo raio no $\mathbb{R}^3$, está área de contato será sempre metade da área total da superfície, ou seja $A=\displaystyle\frac{4\pi r^2}2=2\pi r^2$;
    \item $C_r$ é o Coeficiente de resistência, que se assemelha ao coeficiente de atrito e seu valor depende do formato do objeto. No caso de uma esfera, $C_r=0,47$;
    
\end{itemize}
O código onde essas variáveis são declaradas está logo abaixo, e foi aplicado um fator de conversão para pixels na escala de $1$ cm = $1$ pixel.
\begin{lstlisting}[language=Python]
    rho = 1 #
    coef_ball = 0.47
    A = [2*PI*r*r/10000, 2*PI*bomb_r*bomb_r/10000]
    ball_k = rho*coef_ball*A[0]/2
\end{lstlisting}

A variável A, também armazena a contante $k$ das bombinhas que caem verticalmente.

\subsection*{O Vento e a velocidade relativa}

Em certos níveis de dificuldade do jogo, surge um vento horizontal de sentido arbitrário que pode atrapalhar ou ajudar o usuário. A sua velocidade $v_v$ é dada aleatoriamente nestes níveis.

Para generalizar a modelagem e considerar o impacto que o vento gerará, mudasse apenas uma variável do cálculo da força de retardo. Teremos agora a velocidade relativa,
$$\vec{a}_r=\displaystyle\frac k m\vec{v_r}$$,

onde $\vec{v_r}=v_v-\vec{v}$

OBS.: No código os cálculos de cada componente desses vetores é feito separadamente.

Como inicialmente $v_v=0$, o que ocorre neste caso, é que a fórmula recai em $\vec{a}_r=-\displaystyle\frac k m\vec{v}$

Se a velocidade do vento estiver no mesmo sentido que o movimento, isso reduzirá a força de resistência do ar, ajudando a impulsionar o objeto.
Caso contrário, a resistência do ar será maior, o que retardará ainda mais a bolinha.

\subsection*{Aplicação do Método de Euler}

Para atualizar a posição e aceleração da bola principal, utilizamos o Método de Euler, pois o que temos é aceleração em função da velocidade, além de acelerações e velocidades iniciais. É um problema de equações diferenciais de valor inicial, que optamos por usar a resolução numérica por conta do ciclo do jogo ser rápido.

A velocidade é incrementada a cada frame da seguinte forma:

$\vec{v}_{r(n+1)}=\vec{v}_{r(n)}+\Delta t\cdot  \vec{v}~'_{r(n)}$

$=\vec{v}_{r(n)}+\Delta t\cdot  \vec{a}_{n}$

Onde $\vec{v}_{r(n+1)}$ é a velocidade relativa em dado instante, $\vec{v}_{r(n)}$ é a velocidade relativa no instante anterior.
De forma geral o índice $n+1$ representa o valor da variável atual, e $n$ o seu valor no instante anterior.

$\Delta t$ é um incremento que deve ser um número razoavelmente pequeno que se aproxima de 1 frame.

Após o incremento da velocidade, a aceleração é recalculada da seguinte forma, utilizando as definições de força de resistência e aceleração inicial $\vec{a}_0$:

$\vec{a}_{n+1}=\displaystyle\frac k m\cdot\vec{v}_{r(n+1)}+\vec{a}_0$

Note que a aceleração não usa o Método de Euler diretamente, ela simplesmente aplica a fórmula já deduzida, de resistência do ar, que deve ser reatualizada para garantirmos velocidade variável.

A posição é incrementada pela seguinte fórmula:

$p_{n+1}=p_n+v_{r(n)}\Delta t+\displaystyle\frac{a_{n+1}\Delta t^2}{2}$

Note que a apesar de não estarmos tratando de movimento uniformemente variável, a fórmula da posição é basicamente a mesma do M.U.V. Isso pois no pequeno intervalo de $\Delta t$ podemos por aproximação dizer que o modelo é uniformemente variado, pois neste intervalo mínimo a aceleração é constante.

\subsection*{Ideia Geral do código}
No caso da bola principal, temos dois blocos de códigos que definem valores iniciais: os valores iniciais de repouso, e os valores iniciais no caso de clique do mouse:

\begin{lstlisting}[language=Python]
    ball = 0
    m = 1 #
    r = 20 #
    wind_speed = PVector(0, 0) #
    ball_speed = PVector(0, 0)
    gravity = 9.8 #
    acceleration = PVector(0, gravity)
    
    bomb_list = []
    bomb_acceleration = PVector(0, gravity)
    bomb_r = 40 #
    points = 0
    count = 0
    perdeu = 0
    
    def setup():
        global ball, r
        
        size(900, 900) #
        ball = PVector(r, height - r)
\end{lstlisting}

O bloco de pressionamento da barra de espaços, como dissemos, define uma $\vec{v}_0$ na direção apontada pelo mouse, e aciona a aceleração da gravidade.

\begin{lstlisting}[language=Python]
    def keyReleased():
    global ball, ball_speed, acceleration, t
    if key == ' ':
        if ball.y == height - r:
            ball_speed = (PVector(mouseX, mouseY) - ball)/3 #
            acceleration.y = gravity
\end{lstlisting}
A atualizaçao desses valores conforme as fórmulas explicadas, ocorrem no seguinte bloco:

\begin{lstlisting}[language=Python]
    background(127)
    delta_t = 0.1 #
    count += delta_t
    
    ball_speed.x += acceleration.x*delta_t
    acceleration.x = ball_k*(wind_speed.x - ball_speed.x)/m
    ball_speed.y += acceleration.y*delta_t
    acceleration.y = ball_k*(wind_speed.y - ball_speed.y)/m + gravity
    ball.x += ball_speed.x*delta_t + 0.5*acceleration.x*delta_t*delta_t
    ball.y += ball_speed.y*delta_t + 0.5*acceleration.y*delta_t*delta_t
\end{lstlisting}

A declaração as bombas é feita a cada 100 frames pela função bomb() que quando acionada, coloca na lista de bombas, uma lista de três coordenadas x, y e t, representando a posição x que gerada aleatoriamente entre os limites de tela, a posição y que inicialmente é fora da tela e vai caindo por queda livre, e o tempo que inicialmente e zero e vai sendo incrementado por $\Delta t$.

No mesmo bloco que itera sobre a lista de bombas, ocorre a verificação de colisão com a bola principal, que nada mais é do que verificar se a distância entre os dois objetos coincide com a soma dos dois raios. No caso de colisão, a bomba em questão sai da lista, a bola principal retorna ao canto inferior, os pontos e a fase são incrementados, e a bola principal zera sua velocidade e aceleração para que fique parada no canto inicial. No mesmo bloco há a verificação se a bomba ultrapassou a base inferior da tela, removendo-a da lista e mudando a variável de perda do jogo para verdadeira. Neste caso a bola principal some da tela e surge uma mensagem de "game over". Outra coisa que está no mesmo bloco FOR de atualização das bombas, é o gerador de vento, que a cada fase múltipla de 4, atualiza a velocidade do vento para um aleatório gaussiano. A partir daí as fórmula que calculavam velocidade relativa vão ser colocadas em prática.

\begin{lstlisting}
    def bomb():
    global bomb_list, bomb_r
    bomb_list.append([int(random(bomb_r, width - bomb_r)), -bomb_r, 0])
    
    
    for i in bomb_list:
        i[2] += delta_t
        i[1] = -bomb_r + 0.5*bomb_acceleration.y*i[2]**2
        if ((ball.x - i[0])**2 + (ball.y - i[1])**2)**0.5 < r + bomb_r - 2*(points//5):
            bomb_list.remove(i)
            ball = PVector(r, height - r)
            ball_speed *= 0
            acceleration *= 0
            points += 1
            count = 0
            stage = 1 + points//5
            if stage%4 == 0:
                wind_speed.x = randomGaussian()
            else: wind_speed.x = 0
        if i[1] >= height + bomb_r:
            bomb_list.remove(i)
            perdeu = 1
        noStroke()
        fill(0)
        circle(i[0], i[1], 2*bomb_r - 4*(points//5))
        
        
        if perdeu == 1:
        ball = PVector(-2*r, -2*r, r)
        ball_speed *= 0
        acceleration *= 0
        textSize(125)
        fill(0)
        text('hehe, morreu', 30, height - 30)
        textSize(20)
        text('aperta "r", vai', 30, 750)
\end{lstlisting}



\newpage
\section*{Bibliografia}

Khan Academy. \textit{"Air and Fluid Resistance."}. Acesso em 2 Out. 2019
Disponível em https://www.khanacademy.org/computing/computer-programming/ \\
programming-natural-simulations/programming-forces/a/air-and-fluid-resistance

Processing. \textit{"Processing.py Reference."} Acesso em 2 Out. 2019/\\
Disponível em https://py.processing.org/reference/
\end{document}
